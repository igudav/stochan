\documentclass[12pt, a4paper]{article} % свойства докуменат
\usepackage[utf8]{inputenc} % хотим нормальную кодировку
\usepackage[T2A]{fontenc} % тип шрифта, по-моему
\usepackage[russian]{babel} % русские буквы и обозначения
\usepackage{graphicx, xcolor} % графика
\usepackage{subfiles} % царская разбивка на много файлов
\usepackage{amsmath} % различные нужные символы, типа \geqslant
\usepackage{amssymb} % еще немного символов
\usepackage{bbm}
\usepackage{import} % для включения рисунков
\usepackage{xifthen}
\usepackage{pdfpages}
\usepackage{transparent}
\usepackage{titlesec} % для настройки заголовков и секций вообще
\usepackage{caption} % для подписей к рисункам на 2 строки
\usepackage[outdir=./figures/]{epstopdf}
\usepackage{multicol}
\usepackage{float}
\usepackage{mathrsfs}

% комманда для царского добавления в документ векторной графики
\newcommand{\incfig}[1]{%
    \def\svgwidth{\columnwidth}
    \import{figures/}{#1.pdf_tex}
}
\pdfsuppresswarningpagegroup=1

\newcommand\eqdef{\stackrel{\text{\tiny def}}{=}}

\newtheorem{Th}{Теорема}

% русские знаки нестрогих неравенств
\renewcommand{\le}{\leqslant}
\renewcommand{\ge}{\geqslant}
\renewcommand{\emptyset}{\varnothing}
\renewcommand{\phi}{\varphi}
\renewcommand{\epsilon}{\varepsilon}

\newcommand{\Real}{\mathbb{R}}
\newcommand{\inner}[2]{\bigl< #1, #2 \bigr>}
\newcommand\Set[2]{\left\{ #1 \colon #2 \right\}}
\newcommand\Sum[3]{\sum\limits_{#1 = #2}^{#3}}
\newcommand\Alsur{\mathrel{\stackrel{\mathrm{\text{п.н.}}}=}} % almost shure
\newcommand\Eqtext[1]{\mathrel{\stackrel{\mathrm{\text{#1}}}=}} 

\def\Pro{\mathbb{P}} % вероятность
\def\Ev{\mathcal{F}} % алгебра событий
\def\Bor{\mathscr{B}} % борелевская сигма-алгебра
\def\Real{\mathbb{R}} % вещественная прямая
\def\Int{\mathbb{Z}} % целые числа
\def\Nat{\mathbb{N}} % натуральные числа
\def\Compl{\mathbb{C}} % комплексные числа
\def\Expec{\mathbb{E}} % матожидание 
\def\Ind{\mathbbm{1}} % индикатор
\def\Smpl{\mathbb{X}} % выборка
\def\Med{{\rm med \,}} % медиана
\def\Cov{{\rm cov \,}} % ковыряция
\def\Mod{{\rm mod \,}} % мода
\def\Pois{{\rm Pois \,}} 
\def\Norm{\mathcal{N}} % нормальное распределение

\newcommand*{\hm}[1]{#1\nobreak\discretionary{}%
            {\hbox{\mathsurround=0pt #1}}{}}

\titleformat{\section}{\normalfont\Large\bfseries}{\thesection.}{1em}{}
\titleformat{\subsection}{\normalfont\Large\bfseries}{\thesubsection.}{1em}{}

\DeclareMathOperator{\conv}{conv}
\DeclareMathOperator{\sgn}{sgn}
\DeclareMathOperator{\var}{var}

\newtheorem{St}{Утверждение}
\newtheorem{Def}{Определение}
\newenvironment{Proof}{\par\textbf{Доказательство. }}
	{\hfill$\blacksquare$\vspace{0.1cm}} 

% \counterwithin{section}{part}

\begin{document}

\subfile{titul.tex}

\tableofcontents

\newpage

\section*{Задание 1}

Пусть дана случайная величина $X \sim U[0, 1]$, рассмотрим случайную величину $Y = \Ind(X < p)$. 
Тогда $Y$ принимает значение $1$ с вероятностью $p$.
Это следует из свойства: $\Expec(\Ind_A) = \Pro(A)$.

\begin{Def}
    Биномиальным распределением случайной величины с параметрами $n, p$ 
    будем называть случайную величину, принмающую значения, равные числу успехов в  $n$ испытаниях Бернулли с вероятностью успеха $p$.
\end{Def}
Из определения следует способ генерирования биномиальной случайной величины:
\[
    X = \sum\limits_{i=1}^{n} Y_i, 
\] 
где случайные величины $Y_i$ имеют распределение Бернулли с вероятностью успеха  $p$.

\begin{Def}
    Геометрическим распределением с параметром $p$ (и $q = 1 - p$) будем 
    называть распределение числа неудач до 1-го успеха в схеме Бернулли.
\end{Def}
Функция вероятность геометрического распределения имеет вид: 
$\Pro (X \hm= n) = q^np,\ n \in \Nat_0$.

Пусть случайная величина $Y \sim exp(\lambda)$.
Тогда 
$$
\Pro(n \le Y < n + 1) = e^{-\lambda n} \left(1 - e^{-\lambda}\right),
$$
\[
   X = \lfloor Y \rfloor \sim geom(1 - e^{-\lambda}).
\] 
Тогда чтобы получить геометрическое распределение с параметром $p$ надо 
взять целую часть от експоненциально распределенной случайной величины с параметром  $\lambda = -\ln(1 - p)$
Экспоненциальное распределение получим обращением функции распределения:
если $Z \sim U[0, 1]$, тогда $Y = -\frac{1}{\lambda} \ln(1 - Z) \sim exp(\lambda)$.
Окончательно получаем $X = \left\lfloor \frac{\ln(1 - Z)}{\ln(1 - p)} \right\rfloor\sim geom(p)$.

\begin{St}
    Для $X \sim geom(p)$ верно свойство отсутствия памяти:
    $\Pro (X \ge m + n \mid X \ge m) \hm= \Pro(X \ge n)$.
\end{St}
\begin{Proof}
    \begin{multline*}
        \Pro (X \ge m + n \mid X \ge m) =
        \frac{\Pro(X \ge m + n)}{\Pro(X \ge m)} =\\
        \frac{q^{m + n}p / (1 - q)}{q^{m}p / (1 - q)} = q^{n} =
        \sum\limits_{i=n}^{\infty} q^{i}p = 
        \Pro(X \ge n).
    \end{multline*} 
\end{Proof} 

Рассмотрим игру <<орлянку>> со случайной величиной $X$:
\[
    X_i =
    \begin{cases}
        1\ \text{c вероятностью}\ p = 0{,}5, \\
        -1\ \text{c вероятностью}\ 1 - p = 0{,}5.
    \end{cases} 
\] 
Пусть  $S_n = \sum\limits_{i=1}^{n} X_i$ и $Y = S_i /\!\sqrt{n}$.
Так как $\Expec S_n = 0$ и  $var[S_n] = n$, то из центральной предельной теоремы следует,
что $Y = S_i /\!\sqrt{n} \xrightarrow[n \rightarrow \infty]{d}  \Norm(0, 1)$.

\section*{Задание 2}

\begin{Def}
    Канторовой лестницей нахывается функция $f$, которая строится следующим образом:
    \begin{enumerate}
        \item $f(0) = 0$,  $f(1) = 1$.
        \item Отрезок  $[0, 1]$ разбивается на 3 равных части, затем на 
            средней части функция  $f$ полагается равной полусумме значений на концах.
        \item для первого и третьего отрезка процедура повторяется рекурсивно.
    \end{enumerate} 
\end{Def} 

По определению канторова лестница обладает свойством фрактальности или самоподобия:
$f(x) = 2f(x / 3)$.
Так как  $f$ монотонна и лежит в отрезке  $[0, 1]$, то существует 
случайная величина $X$ (сингулярная), функция распределения которой является канторовой лестницей.

Используя свойство фрактальности данной случайной величины, найдем ее математическое ожидание и дисперсию:
\[
    X \Eqtext{d} 1 - X \implies \Expec X = 0.
\] 
\begin{equation*}
    \Expec X^2 = \int\limits_{0}^{1} x^2 dF(x) = 
    \int\limits_{0}^{\frac{1}{3}} x^2 dF(x) +
    \int\limits_{\frac{2}{3}}^{1}  x^2 dF(x).    
\end{equation*}
Отсюда используя замену $y = 3x$ и свойство $F(x / 3) = F(x) / 2$ получаем 
 \[
     \var\left[ X \right] = \Expec X^2 = \frac{3}{8}.
\] 

Для проверки свойств распределения будем использовать критерии Колмогорова и Смирнова.
В тесте Колмогорова проверяется гипотеза $H_0\colon F = F_0$ соответствия распределения некоторому наперед заданному распределению  $F_0$.
Для проверки гипотезы строится следующая статистика:
 \[
     D_n = \sup\limits_x \bigl\lvert F_n(x) - F_0(x) \bigr\rvert, 
\]
где $F_n(x)$~---~эмпирическая функция распределения, потстроенная по выборке  $X_1, \ldots, X_n$.
Теорема Колмогорова утверждает, что $\sqrt{n}D_n \xrightarrow[n \rightarrow \infty]{d} K$, 
где  $K$~---~распределение Колмогорова.
Зафиксируем некоторый уровень доверия  $\alpha$ и будем отклонять гипотезу, когда
\[
    \sqrt{n} D_n > K^{-1}(1 - \alpha).
\] 

Для проверки гипотезы о принадлежности двух выборок размеров $m$ и  $n$ к одному распределению используется следующая статистика:
 \[
     D_{mn} = \sup\limits_x \bigl\lvert F_n(x) - G_m(x) \bigr\rvert,
\] 
где $F_n$ и  $G_m$~---~эмпирические функции распределения.
По теореме Смирнова 
 \[
     \sqrt{\frac{mn}{m+n}}D_{mn} \xrightarrow[n\rightarrow \infty]{d} K,
\] 
поэтому при больших размерах выборки ($m, n > 20$) отклоняем гипотезу, если 
\[
    \sqrt{\frac{mn}{m+n}}D_{mn} > K^{-1}(1 - \alpha).
\] 

\section*{Задание 3}

\begin{Def}
    Экспоненциальным распределением с параметром $\lambda$ будем называть 
    распределение с плотностью $p(x) = \lambda e^{-\lambda x}$.
\end{Def} 
\begin{St}
    Экспоненциальное распределение обладает свойством отсутствия памяти:
    $$
    \Pro (X \ge x + y \mid X \ge y) \hm= \Pro(X \ge x).
    $$
\end{St} 
\begin{Proof}
    \begin{equation*}
        \Pro (X \ge x + y \mid X \ge y) = 
        \frac{e^{-\lambda (x + y)}}{e^{-\lambda y}} = e^{-\lambda x} = 
        \Pro(X \ge x).
    \end{equation*}
\end{Proof} 

Рассмотрим еще одно свойство экспоненциального распределения.
Пусть $X_1, \ldots , X_n$ нещависимы и распределены экспоненциально с
показателями $\lambda_1,\ldots \lambda_n$ соответственно.
Пусть также $Y = \min \left\{ X_1,\ldots , X_n \right\}$.
Тогда 
\[
    F_Y(x) =\Pro(Y < x) = 
    1 - \Pro\left( X_1 \ge x,\ldots , X_n \ge x \right) =
    1 - \prod\limits_{i=1}^{n} e^{-\lambda_i x} =
    1 - e^{-x\sum\limits_{i=1}^{n} \lambda_i}.
\] 
Таким образом, $Y$ распределена экспоненциально с показателем  $\sum\limits_{i=1}^{n} \lambda_i$.

Рассмотрим два способа моделирования распределения Пуассона.
Первый заключается в том, что промежутки между скачками пуассоновского 
процесса с параметром $\lambda$ распределены экспоненциально, 
а случайная величина  $X(1)$ имеет распределение Пуассона с параметром  $\lambda$.

Второй способ дает 
\begin{Th}[Пуассона]
    Пусть в схеме серий испытаний Бернулли с вероятностями $p_n$ выполнено:
     \[
        p_n \xrightarrow[n \rightarrow\infty]{} 0,\qquad 
        np_n \xrightarrow[n \rightarrow \infty]{} \lambda > 0.
    \]
    Тогда число успехов сходится к распределению Пуассона с параметром $\lambda$.
    При этом верна оценка:
     \[
         \left\lvert \Pro(S_n \in A) - \sum\limits_{k \in A} e^{-\lambda} \frac{\lambda^{k}}{k!}  \right\rvert \le \lambda p.
    \] 
\end{Th}
Чтобы получить вероятность, отличающуюся от распределения Пуассона не больше чем на $\epsilon$,
положим  $p = \frac{\epsilon}{\lambda}$ и $n = \frac{\lambda^2}{\epsilon}$. 
Тогда биномиальное распределение с такими параметрами будет хорошо приближать распределение Пуассона.

Для моделирования стандартного нормального распределения рассмотрим пару случайных величин $(X_1, X_2)$,
которую будем рассматривать, как двумерный вектор.
Пусть  $(r, \phi)$~---~запись вектора в полярных координатах.
Из симметрии $\phi \sim U[0, 2\pi]$.
Из свойств нормального распределения  $r^2 \sim exp(\frac{1}{2})$.
Тогда моделируя угол равномерным распределением и квадрат радиуса экспоненциальным, 
будем получать 2 стандартных нормальных случайных величины после перехода в декартовы координаты.

Критерий $\chi^2$ Пирсона используется для проверки гипотезы равенства распределения заданному $H_0\colon F = F_0$. 
Разобьем прямую на непересекающиеся части $\Delta_1,\ldots \Delta_m$ и
положим $p_i = \Pro(X \in \Delta_i \mid H_0)$.
Обозначим за $N_i$ число элементов выборки, попавших в $\Delta_i$.
Тогда статистика 
 \[
     Q = \sum\limits_{i=1}^{m} \frac{(N_i - np_i)^2}{np_i}
     \sim \chi^2_{m-1}.
\] 
Тогда отклоняем гипотезу, если $\chi^2_{m-1}(Q) < \alpha$.

Пусть дана выборка $X_1,\ldots X_n$ из нормального распределения с неизвестными параметрами.
Для проверки гипотезы $H_0\colon \Expec X = \mu$ используется критерий Стьюдента. 
Положим 
\[
    \hat{\sigma} = \sqrt{\frac{1}{n - 1} \sum\limits_{i=1}^{n} \left( X_i - \overline{X} \right)^2}.
\] 
Тогда статистика 
\[
    U = \sqrt{n} \frac{\overline{X} - \mu}{\hat{\sigma}}
    \sim St_{n-1}.
\] 
Будем отклонять гипотезу, если $St_{n-1}(U) < \frac{\alpha}{2}$.

Для сравнения дисперсий двух нормально распределенных выборок используется критерий Фишера.
Пусть даны две выборки $X_1,\ldots ,X_n$ и  $Y_1,\ldots Y_m$. 
Тогда 
\[
    V = \frac{(m-1)\sum\limits_{i=1}^{n} (X_i - \overline{X})^2}{(n-1)\sum\limits_{i=1}^{m} (Y_i - \overline{Y})^2}
    \sim F_{m-1, n-1}.
\] 
Гипотеза отвергается, если $F_{m-1,n-1}(V) < \frac{\alpha}{2}$.

\section*{Задание 4}



\newpage

\bibliographystyle{utf8gost705u}
\bibliography{biblio}
 \end{document} 
