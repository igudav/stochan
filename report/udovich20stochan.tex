\documentclass[12pt, a4paper]{article} % свойства докуменат
\usepackage[utf8]{inputenc} % хотим нормальную кодировку
\usepackage[T2A]{fontenc} % тип шрифта, по-моему
\usepackage[russian]{babel} % русские буквы и обозначения
\usepackage{graphicx, xcolor} % графика
\usepackage{subfiles} % царская разбивка на много файлов
\usepackage{amsmath} % различные нужные символы, типа \geqslant
\usepackage{amssymb} % еще немного символов
\usepackage{bbm}
\usepackage{import} % для включения рисунков
\usepackage{xifthen}
\usepackage{pdfpages}
\usepackage{transparent}
\usepackage{titlesec} % для настройки заголовков и секций вообще
\usepackage{caption} % для подписей к рисункам на 2 строки
\usepackage[outdir=./figures/]{epstopdf}
\usepackage{multicol}
\usepackage{float}
\usepackage{mathrsfs}

% комманда для царского добавления в документ векторной графики
\newcommand{\incfig}[1]{%
    \def\svgwidth{\columnwidth}
    \import{figures/}{#1.pdf_tex}
}
\pdfsuppresswarningpagegroup=1

\newcommand\eqdef{\stackrel{\text{\tiny def}}{=}}

\newtheorem{Th}{Теорема}

% русские знаки нестрогих неравенств
\renewcommand{\le}{\leqslant}
\renewcommand{\ge}{\geqslant}
\renewcommand{\emptyset}{\varnothing}
\renewcommand{\phi}{\varphi}
\renewcommand{\epsilon}{\varepsilon}

\newcommand{\Real}{\mathbb{R}}
\newcommand{\inner}[2]{\bigl< #1, #2 \bigr>}
\newcommand\Set[2]{\left\{ #1 \colon #2 \right\}}
\newcommand\Sum[3]{\sum\limits_{#1 = #2}^{#3}}
\newcommand\Alsur{\mathrel{\stackrel{\mathrm{\text{п.н.}}}=}} % almost shure
\newcommand\Eqtext[1]{\mathrel{\stackrel{\mathrm{\text{#1}}}=}} 

\def\Pro{\mathbb{P}} % вероятность
\def\Ev{\mathcal{F}} % алгебра событий
\def\Bor{\mathscr{B}} % борелевская сигма-алгебра
\def\Real{\mathbb{R}} % вещественная прямая
\def\Int{\mathbb{Z}} % целые числа
\def\Nat{\mathbb{N}} % натуральные числа
\def\Compl{\mathbb{C}} % комплексные числа
\def\Expec{\mathbb{E}} % матожидание 
\def\Ind{\mathbbm{1}} % индикатор
\def\Smpl{\mathbb{X}} % выборка
\def\Med{{\rm med \,}} % медиана
\def\Cov{{\rm cov \,}} % ковыряция
\def\Mod{{\rm mod \,}} % мода
\def\Pois{{\rm Pois \,}} 
\def\Norm{\mathcal{N}} % нормальное распределение

\newcommand*{\hm}[1]{#1\nobreak\discretionary{}%
            {\hbox{\mathsurround=0pt #1}}{}}

\titleformat{\section}{\normalfont\Large\bfseries}{\thesection.}{1em}{}
\titleformat{\subsection}{\normalfont\Large\bfseries}{\thesubsection.}{1em}{}

\DeclareMathOperator{\conv}{conv}
\DeclareMathOperator{\sgn}{sgn}
\DeclareMathOperator{\var}{var}

\newtheorem{St}{Утверждение}
\newtheorem{Def}{Определение}
\newenvironment{Proof}{\par\textbf{Доказательство. }}
	{\hfill$\blacksquare$\vspace{0.1cm}} 

% \counterwithin{section}{part}

\begin{document}

\subfile{titul.tex}

\tableofcontents

\newpage

\section*{Задание 1}

Пусть дана случайная величина $X \sim U[0, 1]$, рассмотрим случайную величину $Y = \Ind(X < p)$. 
Тогда $Y$ принимает значение $1$ с вероятностью $p$.
Это следует из свойства: $\Expec(\Ind_A) = \Pro(A)$.

\begin{Def}
    Биномиальным распределением случайной величины с параметрами $n, p$ 
    будем называть случайную величину, принмающую значения, равные числу успехов в  $n$ испытаниях Бернулли с вероятностью успеха $p$.
\end{Def}
Из определения следует способ генерирования биномиальной случайной величины:
\[
    X = \sum\limits_{i=1}^{n} Y_i, 
\] 
где случайные величины $Y_i$ имеют распределение Бернулли с вероятностью успеха  $p$.

\begin{Def}
    Геометрическим распределением с параметром $p$ (и $q = 1 - p$) будем 
    называть распределение числа неудач до 1-го успеха в схеме Бернулли.
\end{Def}
Функция вероятность геометрического распределения имеет вид: 
$\Pro (X \hm= n) = q^np,\ n \in \Nat_0$.

Пусть случайная величина $Y \sim exp(\lambda)$.
Тогда 
$$
\Pro(n \le Y < n + 1) = e^{-\lambda n} \left(1 - e^{-\lambda}\right),
$$
\[
   X = \lfloor Y \rfloor \sim geom(1 - e^{-\lambda}).
\] 
Тогда чтобы получить геометрическое распределение с параметром $p$ надо 
взять целую часть от експоненциально распределенной случайной величины с параметром  $\lambda = -\ln(1 - p)$
Экспоненциальное распределение получим обращением функции распределения:
если $Z \sim U[0, 1]$, тогда $Y = -\frac{1}{\lambda} \ln(1 - Z) \sim exp(\lambda)$.
Окончательно получаем $X = \left\lfloor \frac{\ln(1 - Z)}{\ln(1 - p)} \right\rfloor\sim geom(p)$.

\begin{St}
    Для $X \sim geom(p)$ верно свойство отсутствия памяти:
    $\Pro (X \ge m + n \mid X \ge m) \hm= \Pro(X \ge n)$.
\end{St}
\begin{Proof}
    \begin{multline*}
        \Pro (X \ge m + n \mid X \ge m) =
        \frac{\Pro(X \ge m + n)}{\Pro(X \ge m)} =\\
        \frac{q^{m + n}p / (1 - q)}{q^{m}p / (1 - q)} = q^{n} =
        \sum\limits_{i=n}^{\infty} q^{i}p = 
        \Pro(X \ge n).
    \end{multline*} 
\end{Proof} 

Рассмотрим игру <<орлянку>> со случайной величиной $X$:
\[
    X_i =
    \begin{cases}
        1\ \text{c вероятностью}\ p = 0{,}5, \\
        -1\ \text{c вероятностью}\ 1 - p = 0{,}5.
    \end{cases} 
\] 
Пусть  $S_n = \sum\limits_{i=1}^{n} X_i$ и $Y = S_i /\!\sqrt{n}$.
Так как $\Expec S_n = 0$ и  $var[S_n] = n$, то из центральной предельной теоремы следует,
что $Y = S_i /\!\sqrt{n} \xrightarrow[n \rightarrow \infty]{d}  \Norm(0, 1)$.

\section*{Задание 2}

\begin{Def}
    Канторовой лестницей нахывается функция $f$, которая строится следующим образом:
    \begin{enumerate}
        \item $f(0) = 0$,  $f(1) = 1$.
        \item Отрезок  $[0, 1]$ разбивается на 3 равных части, затем на 
            средней части функция  $f$ полагается равной полусумме значений на концах.
        \item для первого и третьего отрезка процедура повторяется рекурсивно.
    \end{enumerate} 
\end{Def} 

По определению канторова лестница обладает свойством фрактальности или самоподобия:
$f(x) = 2f(x / 3)$.
Так как  $f$ монотонна и лежит в отрезке  $[0, 1]$, то существует 
случайная величина $X$ (сингулярная), функция распределения которой является канторовой лестницей.

Используя свойство фрактальности данной случайной величины, найдем ее математическое ожидание и дисперсию:
\[
    X \Eqtext{d} 1 - X \implies \Expec X = 0.
\] 
\begin{equation*}
    \Expec X^2 = \int\limits_{0}^{1} x^2 dF(x) = 
    \int\limits_{0}^{\frac{1}{3}} x^2 dF(x) +
    \int\limits_{\frac{2}{3}}^{1}  x^2 dF(x).    
\end{equation*}
Отсюда используя замену $y = 3x$ и свойство $F(x / 3) = F(x) / 2$ получаем 
 \[
     \var\left[ X \right] = \Expec X^2 = \frac{3}{8}.
\] 

\newpage

\bibliographystyle{utf8gost705u}
\bibliography{biblio}
 \end{document} 
